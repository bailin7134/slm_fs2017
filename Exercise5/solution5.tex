% --------------------------------------------------------------
% This is all preamble stuff that you don't have to worry about.
% Head down to where it says "Start here"
% --------------------------------------------------------------

\documentclass[12pt]{article}

\usepackage[margin=1in]{geometry} 
\usepackage{amsmath,amsthm,amssymb}
\usepackage{color}
%\usepackage{tikz, pgfplots}
\usepackage{graphicx}
\usepackage{epstopdf} %converting to PDF
\usepackage{subcaption}
\usepackage{listings}

\makeatletter

\renewcommand\section{\@startsection {section}{1}{\z@}%
	{-3.5ex \@plus -1ex \@minus -.2ex}%
	{2.3ex \@plus.2ex}%
	{\normalfont\large\bfseries}}% from \Large
\renewcommand\subsection{\@startsection{subsection}{2}{\z@}%
	{-3.25ex\@plus -1ex \@minus -.2ex}%
	{1.5ex \@plus .2ex}%
	{\normalfont\large\bfseries}}% from \large
\makeatother

\begin{document}
	
	% --------------------------------------------------------------
	%                         Start here
	% --------------------------------------------------------------
	
	%\renewcommand{\qedsymbol}{\filledbox}
	
	\title{\textbf{Statistical Learning Methods Exercise \#{5}}\\
	Universit{\'e} de Neuch\^{a}tel}%replace X with the appropriate number
	\author{{Lin Bai, 09935404}} %replace with your name
	
	\maketitle

	%%%%%%%%%%%%%%%%%%%%%%%%%%%%%%%%%%%%%%%%%%%%%%%%%%%%%%%%%%%%%%%%%%%%%%%%%%%%%%%%%%%
	%%%%%%   question 1
	%%%%%%%%%%%%%%%%%%%%%%%%%%%%%%%%%%%%%%%%%%%%%%%%%%%%%%%%%%%%%%%%%%%%%%%%%%%%%%%%%%%
	\section{Solution to question 1}
	The multiple R-squared is 0.9931, which means the variables "Education" and "Gender" have a strong relation to "Wage".\\
	
	\lstset{language=R}
	\lstset{frame=lines}
	\lstset{label={lst:code_direct}}
	\lstset{basicstyle=\footnotesize\ttfamily}
	\begin{lstlisting}[breaklines=true]
		Call:
		lm(formula = Wage ~ Education + Gender, data = eduData)
	 
		Residuals:
		  Min       1Q      Median    3Q      Max 
		-243.473  -76.073    0.354   73.126  280.275 
	 
		Coefficients:
		            Estimate  Std. Error  t value Pr(>|t|)    
		(Intercept) -569.784     23.100  -24.67   <2e-16 ***
		Education    397.975      1.543  258.00   <2e-16 ***
		Gendermale   597.904      9.173   65.18   <2e-16 ***
		---
		Signif. codes:  0 ‘***’ 0.001 ‘**’ 0.01 ‘*’ 0.05 ‘.’ 0.1 ‘ ’ 1
	 
		Residual standard error: 100.1 on 494 degrees of freedom
		Multiple R-squared:  0.9931,	Adjusted R-squared:  0.9931 
		F-statistic: 3.544e+04 on 2 and 494 DF,  p-value: < 2.2e-16
	\end{lstlisting}
	%\begin{figure}[htbp]
	%	\centering
	%	\includegraphics[width=0.75\textwidth]{q1.eps}
	%	\caption{education wage plot}
	%\end{figure}
	\newpage
	%%%%%%%%%%%%%%%%%%%%%%%%%%%%%%%%%%%%%%%%%%%%%%%%%%%%%%%%%%%%%%%%%%%%%%%%%%%%%%%%%%%
	%%%%%%   question 2
	%%%%%%%%%%%%%%%%%%%%%%%%%%%%%%%%%%%%%%%%%%%%%%%%%%%%%%%%%%%%%%%%%%%%%%%%%%%%%%%%%%%
	\section{Solution to question 2}
	\subsection{Question 2-1}
	parameters "vendor" and "model" are not suitable to predict performance.
	\subsection{Question 2-2}
	\lstset{language=R}
	\lstset{frame=lines}
	\lstset{label={lst:code_direct}}
	\lstset{basicstyle=\footnotesize\ttfamily}
	\begin{lstlisting}[breaklines=true]
		Call:
		lm(formula = PRP ~ MYCT + MMIN + MMAX + CACH + CGMIN + CHMAX, 
		data = pcData)
	
		Residuals:
		  Min      1Q     Median   3Q     Max 
		-195.82  -25.17    5.40   26.52  385.75 
	 
		Coefficients:
	                 Estimate   Std. Error t value Pr(>|t|)    
		(Intercept) -5.589e+01  8.045e+00  -6.948 5.00e-11 ***
		MYCT         4.885e-02  1.752e-02   2.789   0.0058 ** 
		MMIN         1.529e-02  1.827e-03   8.371 9.42e-15 ***
		MMAX         5.571e-03  6.418e-04   8.681 1.32e-15 ***
		CACH         6.414e-01  1.396e-01   4.596 7.59e-06 ***
		CGMIN       -2.704e-01  8.557e-01  -0.316   0.7524    
		CHMAX        1.482e+00  2.200e-01   6.737 1.65e-10 ***
		---
		Signif. codes:  0 ‘***’ 0.001 ‘**’ 0.01 ‘*’ 0.05 ‘.’ 0.1 ‘ ’ 1
	 
		Residual standard error: 59.99 on 202 degrees of freedom
		Multiple R-squared:  0.8649,	Adjusted R-squared:  0.8609 
		F-statistic: 215.5 on 6 and 202 DF,  p-value: < 2.2e-16
	\end{lstlisting}
	\subsection{Question 2-3}
	Parameter "MYCT" and "CGMIN" have very little contribution on the model we build.
	\subsection{Question 2-4}
	The method is:\\
	1. get the adjusted R square for each parameter\\
	2. choose the parameter with biggest adjusted R square to build the model\\
	3. add one of the not used parameters to the model and check the adjusted R square of
	new model, if none of them is greater than the reference model, then stop. Otherwise, 
	choose current model as reference model\\
	3. re-run step 2 until it stops.
	\subsection{Question 2-5}
	$Pr(>|t|)$ demonstrates how strong each parameter has influence on the final model\\
	Multiple R-squared shows how good the model is to predict the performance
	%%%%%%%%%%%%%%%%%%%%%%%%%%%%%%%%%%%%%%%%%%%%%%%%%%%%%%%%%%%%%%%%%%%%%%%%%%%%%%%%%%%
	%%%%%%   question 3
	%%%%%%%%%%%%%%%%%%%%%%%%%%%%%%%%%%%%%%%%%%%%%%%%%%%%%%%%%%%%%%%%%%%%%%%%%%%%%%%%%%%
	\section{Solution to question 3}
	As shown in the figures, the outliers are No. 10, No. 32, No. 200.\\
	\begin{figure}[htbp]
		\centering
		\includegraphics[width=0.75\textwidth]{3-1.eps}
		\caption{education wage plot}
	\end{figure}
	\newpage
	%%%%%%%%%%%%%%%%%%%%%%%%%%%%%%%%%%%%%%%%%%%%%%%%%%%%%%%%%%%%%%%%%%%%%%%%%%%%%%%%%%%
	%%%%%%   question 4
	%%%%%%%%%%%%%%%%%%%%%%%%%%%%%%%%%%%%%%%%%%%%%%%%%%%%%%%%%%%%%%%%%%%%%%%%%%%%%%%%%%%
	\section{Solution to question 4}
	Repeat the Question 2 and 3\\
	
	\lstset{language=R}
	\lstset{frame=lines}
	\lstset{label={lst:code_direct}}
	\lstset{basicstyle=\footnotesize\ttfamily}
	\begin{lstlisting}[breaklines=true]
		Call:
		lm(formula = mpg ~ weight + year + horsepower + acceleration + 
		    displacement + cylinders, data = carData)
		
		Residuals:
		    Min      1Q  Median      3Q     Max 
		-8.6927 -2.3864 -0.0801  2.0291 14.3607 
		
		Coefficients:
					   Estimate Std. Error t value Pr(>|t|)    
		(Intercept)  -1.454e+01  4.764e+00  -3.051  0.00244 ** 
		weight       -6.795e-03  6.700e-04 -10.141  < 2e-16 ***
		year          7.534e-01  5.262e-02  14.318  < 2e-16 ***
		horsepower   -3.914e-04  1.384e-02  -0.028  0.97745    
		acceleration  8.527e-02  1.020e-01   0.836  0.40383    
		displacement  7.678e-03  7.358e-03   1.044  0.29733    
		cylinders    -3.299e-01  3.321e-01  -0.993  0.32122    
		---
		Signif. codes:  0 ‘***’ 0.001 ‘**’ 0.01 ‘*’ 0.05 ‘.’ 0.1 ‘ ’ 1
		
		Residual standard error: 3.435 on 385 degrees of freedom
		  (6 observations deleted due to missingness)
		Multiple R-squared:  0.8093,	Adjusted R-squared:  0.8063 
		F-statistic: 272.2 on 6 and 385 DF,  p-value: < 2.2e-16
	\end{lstlisting}
	\noindent
	\\
	As shown in the figures, the outliers are No. 323, No. 326, No. 327.\\
	\begin{figure}[htbp]
		\centering
		\includegraphics[width=0.75\textwidth]{5-1.eps}
		\caption{education wage plot}
	\end{figure}
	% --------------------------------------------------------------
	%     You don't have to mess with anything below this line.
	% --------------------------------------------------------------
	
\end{document}