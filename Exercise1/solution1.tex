% --------------------------------------------------------------
% This is all preamble stuff that you don't have to worry about.
% Head down to where it says "Start here"
% --------------------------------------------------------------

\documentclass[12pt]{article}

\usepackage[margin=1in]{geometry} 
\usepackage{amsmath,amsthm,amssymb}
\usepackage{color}
%\usepackage{tikz, pgfplots}
\usepackage{graphicx}
\usepackage{epstopdf} %converting to PDF
\usepackage{subcaption}
\usepackage{listings}

\makeatletter

\renewcommand\section{\@startsection {section}{1}{\z@}%
	{-3.5ex \@plus -1ex \@minus -.2ex}%
	{2.3ex \@plus.2ex}%
	{\normalfont\large\bfseries}}% from \Large
\renewcommand\subsection{\@startsection{subsection}{2}{\z@}%
	{-3.25ex\@plus -1ex \@minus -.2ex}%
	{1.5ex \@plus .2ex}%
	{\normalfont\large\bfseries}}% from \large
\makeatother

\begin{document}
	
	% --------------------------------------------------------------
	%                         Start here
	% --------------------------------------------------------------
	
	%\renewcommand{\qedsymbol}{\filledbox}
	
	\title{\textbf{Statistical Learning Methods Exercise \#{1}}\\
	Universit{\'e} de Neuch\^{a}tel}%replace X with the appropriate number
	\author{{Lin Bai, 09935404}} %replace with your name
	
	\maketitle

	%%%%%%%%%%%%%%%%%%%%%%%%%%%%%%%%%%%%%%%%%%%%%%%%%%%%%%%%%%%%%%%%%%%%%%%%%%%%%%%%%%%
	%%%%%%   question 1
	%%%%%%%%%%%%%%%%%%%%%%%%%%%%%%%%%%%%%%%%%%%%%%%%%%%%%%%%%%%%%%%%%%%%%%%%%%%%%%%%%%%
	\section{Solution to question 1}
	If negative item is removed, the result is shown bellow:
	\lstset{language=R}
	\lstset{frame=lines}
	\lstset{caption={Filter out wrong education and gender}}
	\lstset{label={lst:code_direct}}
	\lstset{basicstyle=\footnotesize\ttfamily}
	\begin{lstlisting}[breaklines=true]
	# filter all negative value in education
	educationDataFrame <- read.table("Education.txt", header=TRUE, sep="")
	
	iter = 1
	val = 1
	data_length = length(educationDataFrame$ID)
	
	while(iter <= data_length)
	{
		# filter wrong education & gender
		if (educationDataFrame$Education[val]<0)
		{
			educationDataFrame = educationDataFrame[-c(val), ]
		}
		else if (educationDataFrame$Gender[val] != 1 & educationDataFrame$Gender[val] != 2)
		{
			educationDataFrame = educationDataFrame[-c(val), ]
		}
		else
			val = val + 1
	
		iter = iter + 1
	}
	\end{lstlisting}
	
	To get the mean, the median, the standard deviation, the minimum and maximum value.
	\lstset{caption={Summary of each variable}}
	\begin{lstlisting}[breaklines=true]
		summary(educationDataFrame)
		
		sd(educationDataFrame$Education)
		sd(educationDataFrame$Gender)
		sd(educationDataFrame$Wage)
		sqrt(mean(educationDataFrame$Education^2) - mean(educationDataFrame$Education)^2)
		sqrt(mean(educationDataFrame$Gender^2) - mean(educationDataFrame$Gender)^2)
		sqrt(mean(educationDataFrame$Wage^2) - mean(educationDataFrame$Wage)^2)
	\end{lstlisting}
	\begin{table}[htbp]
		\centering
		\label{my-label}
		\begin{tabular}{l|l|l|l|l|l|l}
			\hline
			          & mean   & median & minimum & maximum & sample sd & population sd \\
			\hline
			education & 14.26  & 14.00  & 5.00    & 22.00   & 2.912     & 2.909\\
			\hline
			gender    & 1.4    & 1.0    & 1.0     & 2.0     & 0.490     & 0.490\\
			\hline
			wage      & 5468.4 & 5515.7 & 41.8    & 8453.5  & 1223.807  & 1222.578\\
			\hline
		\end{tabular}
	\end{table}
	
	%%%%%%%%%%%%%%%%%%%%%%%%%%%%%%%%%%%%%%%%%%%%%%%%%%%%%%%%%%%%%%%%%%%%%%%%%%%%%%%%%%%
	%%%%%%   question 2
	%%%%%%%%%%%%%%%%%%%%%%%%%%%%%%%%%%%%%%%%%%%%%%%%%%%%%%%%%%%%%%%%%%%%%%%%%%%%%%%%%%%
	\section{Solution to question 2}
	\subsection{Country: US}
	\lstset{caption={US}}
	\begin{lstlisting}[breaklines=true]
		usSub<-subset(educationDataFrame, educationDataFrame$Country=="US")
		usDataFrame<-subset(usSub, usSub$Education>0 & (usSub$Gender==1 | usSub$Gender==2))
		summary(usDataFrame)
	\end{lstlisting}
		\begin{table}[htbp]
		\centering
		\label{my-label}
		\begin{tabular}{l|l|l|l|l|l|l}
			\hline
			          & mean   & median & minimum & maximum & sample sd & population sd \\
			\hline
			education & 14.19  & 14.00   & 8.00    & 21.00   & 2.782     & 2.777\\
			\hline
			gender    & 1.396  & 1.0    & 1.0     & 2.0     & 0.490     & 0.4890\\
			\hline
			wage      & 5459   & 5447   & 2938    & 8110    & 1169.657  & 1167.693\\
			\hline
		\end{tabular}
	\end{table}

	
	\subsection{Country: Canda}
	\lstset{caption={Canda}}
	\begin{lstlisting}[breaklines=true]
		caSub<-subset(educationDataFrame, educationDataFrame$Country=="Canada")
		caDataFrame<-subset(caSub, caSub$Education>0 & (caSub$Gender==1 | caSub$Gender==2))
		summary(caDataFrame)
	\end{lstlisting}
	\begin{table}[htbp]
		\centering
		\label{my-label}
		\begin{tabular}{l|l|l|l|l|l|l}
			\hline
			          & mean   & median & minimum & maximum & sample sd & population sd \\
			\hline
			education & 14.36  & 14.00  & 5.00    & 22.00   & 3.100    & 3.092\\
			\hline
			gender    & 1.405  & 1.0    & 1.0     & 2.0     & 0.492     & 0.491\\
			\hline
			wage      & 5482.1 & 5634.2 & 41.8    & 8453.5  & 1303.207  & 1299.945\\
			\hline
		\end{tabular}
	\end{table}
	%%%%%%%%%%%%%%%%%%%%%%%%%%%%%%%%%%%%%%%%%%%%%%%%%%%%%%%%%%%%%%%%%%%%%%%%%%%%%%%%%%%
	%%%%%%   question 3
	%%%%%%%%%%%%%%%%%%%%%%%%%%%%%%%%%%%%%%%%%%%%%%%%%%%%%%%%%%%%%%%%%%%%%%%%%%%%%%%%%%%
	\section{Solution to question 3}
	From the calculation in last section. We could find, the minimum wage is Canada is lower than US, while the maximum wage in Canada is higher then that in US. Since the standard derivation in Canada is about 200 higher than that in US, we could say the wage in US has less difference between each other.\\
	%%%%%%%%%%%%%%%%%%%%%%%%%%%%%%%%%%%%%%%%%%%%%%%%%%%%%%%%%%%%%%%%%%%%%%%%%%%%%%%%%%%
	%%%%%%   question 4
	%%%%%%%%%%%%%%%%%%%%%%%%%%%%%%%%%%%%%%%%%%%%%%%%%%%%%%%%%%%%%%%%%%%%%%%%%%%%%%%%%%%
	\section{Solution to question 4}
	\subsection{male}
	\lstset{caption={male}}
	\begin{lstlisting}[breaklines=true]
		maleSub<-subset(educationDataFrame, educationDataFrame$Gender=="1")
		maleDataFrame<-subset(maleSub, maleSub$Education>0 & (maleSub$Gender==1 | maleSub$Gender==2))
		summary(maleDataFrame)
	\end{lstlisting}
	\begin{table}[htbp]
		\centering
		\label{my-label}
		\begin{tabular}{l|l|l|l|l|l|l}
			\hline
			          & mean   & median & minimum & maximum & sample sd & population sd \\
			\hline
			education & 14.26  & 14.00  & 5.00    & 21.00   & 2.941     & 2.936\\
			\hline
			wage      & 5730   & 5730   & 2047    & 8454    & 1168.759  & 1166.802\\
			\hline
		\end{tabular}
	\end{table}

	\subsection{female}
	\lstset{caption={male}}
	\begin{lstlisting}[breaklines=true]
		femaleSub<-subset(educationDataFrame, educationDataFrame$Gender=="2")
		femaleDataFrame<-subset(femaleSub, femaleSub$Education>0 & (femaleSub$Gender==1 | femaleSub$Gender==2))
		summary(femaleDataFrame)
	\end{lstlisting}
	\begin{table}[htbp]
	\centering
	\label{my-label}
	\begin{tabular}{l|l|l|l|l|l|l}
		\hline
	              & mean   & median & minimum & maximum & sample sd & population sd \\
		\hline
		education & 14.25  & 14.00  & 8.00    & 22.00   & 2.874     & 2.867\\
		\hline
		wage      & 5074.9 & 4951.9 & 41.8    & 8329    & 1201.782  & 1198.759\\
		\hline
	\end{tabular}
	\end{table}
	\noindent
	From the result above, we could find, both the min. and max. education year of female is larger then that of male. And less difference in education year among female. Concerning to the wage, female has higher average wave but larger difference among female.\\
	

	
	% --------------------------------------------------------------
	%     You don't have to mess with anything below this line.
	% --------------------------------------------------------------
	
\end{document}