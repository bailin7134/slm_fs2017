% --------------------------------------------------------------
% This is all preamble stuff that you don't have to worry about.
% Head down to where it says "Start here"
% --------------------------------------------------------------

\documentclass[12pt]{article}

\usepackage[margin=1in]{geometry} 
\usepackage{amsmath,amsthm,amssymb}
\usepackage{color}
%\usepackage{tikz, pgfplots}
\usepackage{graphicx}
\usepackage{epstopdf} %converting to PDF
\usepackage{subcaption}
\usepackage{listings}

\makeatletter

\renewcommand\section{\@startsection {section}{1}{\z@}%
	{-3.5ex \@plus -1ex \@minus -.2ex}%
	{2.3ex \@plus.2ex}%
	{\normalfont\large\bfseries}}% from \Large
\renewcommand\subsection{\@startsection{subsection}{2}{\z@}%
	{-3.25ex\@plus -1ex \@minus -.2ex}%
	{1.5ex \@plus .2ex}%
	{\normalfont\large\bfseries}}% from \large
\makeatother

\begin{document}
	
	% --------------------------------------------------------------
	%                         Start here
	% --------------------------------------------------------------
	
	%\renewcommand{\qedsymbol}{\filledbox}
	
	\title{\textbf{Statistical Learning Methods Exercise \#{8}}\\
	Universit{\'e} de Neuch\^{a}tel}%replace X with the appropriate number
	\author{{Lin Bai, 09935404}} %replace with your name
	
	\maketitle

	%%%%%%%%%%%%%%%%%%%%%%%%%%%%%%%%%%%%%%%%%%%%%%%%%%%%%%%%%%%%%%%%%%%%%%%%%%%%%%%%%%%
	%%%%%%   question 1
	%%%%%%%%%%%%%%%%%%%%%%%%%%%%%%%%%%%%%%%%%%%%%%%%%%%%%%%%%%%%%%%%%%%%%%%%%%%%%%%%%%%
	\section{Solution to question 1}
	\subsection{}
	Three linear regression models, single linear regression, multi linear regression and one using polynomial.\\
	\lstset{language=R}
	\lstset{frame=lines}
	\lstset{label={lst:code_direct}}
	\lstset{basicstyle=\footnotesize\ttfamily}
	\begin{lstlisting}[breaklines=true]
	carData <- read.table("Cars2Data.txt",header=T,sep="")
	# useful data extraction
	usefulData = carData[, c("mpg", "cylinders", "displacement", "horsepower", "weight", "acceleration", "year")]
	usefulData<-subset(usefulData, horsepower>0)
	predictors = names(usefulData)
	# Standardized the values
	means = lapply(usefulData, mean)
	sd = lapply(usefulData, sd)
	usefulData = (usefulData - means) / sd
	summary(usefulData)
	attach(usefulData)
	
	trainSize <- floor(0.8*nrow(usefulData))
	set.seed(123)
	trainIndex <- sample(seq_len(nrow(usefulData)), size=trainSize)
	trainSet <- usefulData[trainIndex,]
	testSet <- usefulData[-trainIndex,]
	
	
	# model 1: simple linear regression
	library(boot)
	print("signle regression time")
	glm.mod1 <- glm(mpg~weight, data=trainSet)
	cv.errorSigReg <- cv.glm(trainSet,glm.mod1,K=10)$delta[2]
	cv.errorSigReg
	# model 2: multi linear regression
	glm.mod2 <- glm(mpg~weight+year+horsepower+acceleration+displacement+cylinders, data=trainSet)
	cv.errorMul1Reg <- cv.glm(trainSet,glm.mod2,K=10)$delta[2]
	cv.errorMul1Reg
	# model 3: multiple linear regression
	cv.errorPolyReg <- rep(0,10)
	for (i in 1:10) {
		glm.mod3 <- glm(mpg~poly(weight,i), data=trainSet)
		cv.errorPolyReg[i] <- cv.glm(trainSet,glm.mod3,K=10)$delta[2]
	}
	cv.errorPolyReg
	# the best model is i=2
	glm.mod3 <- glm(mpg~poly(weight,2), data=trainSet)
	\end{lstlisting}
	When using polynomial as predictors, weight's square gives the lowest error.\\
	\subsection{}
	10-fold cross validation is also run in above code. The test errors are 0.3084253, 0.1957458 and 0.2868424 for single, multi and polynomial respectively.\\
	\subsection{}
	T-test results are shown below, p-values are not as small as we could accept the hypothesis.\\
	\lstset{language=R}
	\lstset{frame=lines}
	\lstset{label={lst:code_direct}}
	\lstset{basicstyle=\footnotesize\ttfamily}
	\begin{lstlisting}[breaklines=true]
	> t.test(predict(glm.mod1, data=testSet),testSet$mpg)
	
	Welch Two Sample t-test
	
	data:  predict(glm.mod1, data = testSet) and testSet$mpg
	t = 0.9135, df = 117.62, p-value = 0.3628
	alternative hypothesis: true difference in means is not equal to 0
	95 percent confidence interval:
	-0.1181945  0.3206087
	sample estimates:
	mean of x   mean of y 
	0.02039632 -0.08081075 
	
	> t.test(predict(glm.mod2, data=testSet),testSet$mpg)
	
	Welch Two Sample t-test
	
	data:  predict(glm.mod2, data = testSet) and testSet$mpg
	t = 0.89964, df = 124.43, p-value = 0.3701
	alternative hypothesis: true difference in means is not equal to 0
	95 percent confidence interval:
	-0.1214496  0.3238637
	sample estimates:
	mean of x   mean of y 
	0.02039632 -0.08081075 
	
	> t.test(predict(glm.mod3, data=testSet),testSet$mpg)
	
	Welch Two Sample t-test
	
	data:  predict(glm.mod3, data = testSet) and testSet$mpg
	t = 0.91084, df = 118.9, p-value = 0.3642
	alternative hypothesis: true difference in means is not equal to 0
	95 percent confidence interval:
	-0.1188117  0.3212259
	sample estimates:
	mean of x   mean of y 
	0.02039632 -0.08081075  
	\end{lstlisting}
	%%%%%%%%%%%%%%%%%%%%%%%%%%%%%%%%%%%%%%%%%%%%%%%%%%%%%%%%%%%%%%%%%%%%%%%%%%%%%%%%%%%
	%%%%%%   question 2
	%%%%%%%%%%%%%%%%%%%%%%%%%%%%%%%%%%%%%%%%%%%%%%%%%%%%%%%%%%%%%%%%%%%%%%%%%%%%%%%%%%%
	\section{Solution to question 2}
	\lstset{language=R}
	\lstset{frame=lines}
	\lstset{label={lst:code_direct}}
	\lstset{basicstyle=\footnotesize\ttfamily}
	\begin{lstlisting}[breaklines=true]
	cancerData <- read.table("Cancer.txt",header=T,sep="")
	# useful data extraction
	usefulData = cancerData[, c("Radius","Texture","Perimeter","Area","Smooth","Compact","Concavity","Concave","Symmetry","Fractal")]
	predictors = names(usefulData)
	# Standardized the values
	means = lapply(usefulData, mean)
	sd = lapply(usefulData, sd)
	usefulData = (usefulData - means) / sd
	#featData = usefulData
	usefulData[, "Diagnostic"] = cancerData[, "Diagnostic"]
	summary(usefulData)
	attach(usefulData)
	
	# perform logistic regression
	glm.mod <- glm(Diagnostic~., data=usefulData, family=binomial)
	summary(glm.mod)
	# most important value is "texture"
	
	glm.mod <- glm(Diagnostic~Texture, data=usefulData, family=binomial)
	summary(glm.mod)
	glm.probs <- predict(glm.mod, type="response")
	glm.pred = rep("B",dim(usefulData)[1])
	glm.pred[glm.probs >.5] = "M"
	table(glm.pred,Diagnostic)
	mean(glm.pred==Diagnostic)
	\end{lstlisting}
	From the summary result, below. The most important value is "texture". And the error rate is 1-0.7047452=0.2952548.\\ 
	\lstset{language=R}
	\lstset{frame=lines}
	\lstset{label={lst:code_direct}}
	\lstset{basicstyle=\footnotesize\ttfamily}
	\begin{lstlisting}[breaklines=true]
	Call:
	glm(formula = Diagnostic ~ ., family = binomial, data = usefulData)
	
	Deviance Residuals: 
	Min        1Q    Median        3Q       Max  
	-1.95590  -0.14839  -0.03943   0.00429   2.91690  
	
	Coefficients:
	Estimate Std. Error z value Pr(>|z|)    
	(Intercept)  0.48702    0.56432   0.863   0.3881    
	Radius      -7.22185   13.09494  -0.551   0.5813    
	Texture      1.65476    0.27758   5.961  2.5e-09 ***
	Perimeter   -1.73763   12.27499  -0.142   0.8874    
	Area        14.00485    5.89090   2.377   0.0174 *  
	Smooth       1.07495    0.44942   2.392   0.0168 *  
	Compact     -0.07723    1.07434  -0.072   0.9427    
	Concavity    0.67512    0.64733   1.043   0.2970    
	Concave      2.59287    1.10701   2.342   0.0192 *  
	Symmetry     0.44626    0.29143   1.531   0.1257    
	Fractal     -0.48248    0.60406  -0.799   0.4244    
	---
	Signif. codes:  0 ‘***’ 0.001 ‘**’ 0.01 ‘*’ 0.05 ‘.’ 0.1 ‘ ’ 1
	
	(Dispersion parameter for binomial family taken to be 1)
	
	Null deviance: 751.44  on 568  degrees of freedom
	Residual deviance: 146.13  on 558  degrees of freedom
	AIC: 168.13
	
	Number of Fisher Scoring iterations: 9
	
	
	> glm.probs <- predict(glm.mod, type="response")
	> glm.pred = rep("B",dim(usefulData)[1])
	> glm.pred[glm.probs >.5] = "M"
	> table(glm.pred,Diagnostic)
	Diagnostic
	glm.pred   B   M
	B 310 121
	M  47  91
	> mean(glm.pred==Diagnostic)
	[1] 0.7047452
	\end{lstlisting}
	%%%%%%%%%%%%%%%%%%%%%%%%%%%%%%%%%%%%%%%%%%%%%%%%%%%%%%%%%%%%%%%%%%%%%%%%%%%%%%%%%%%
	%%%%%%   question 3
	%%%%%%%%%%%%%%%%%%%%%%%%%%%%%%%%%%%%%%%%%%%%%%%%%%%%%%%%%%%%%%%%%%%%%%%%%%%%%%%%%%%
	\section{Solution to question 3}
	\lstset{language=R}
	\lstset{frame=lines}
	\lstset{label={lst:code_direct}}
	\lstset{basicstyle=\footnotesize\ttfamily}
	\begin{lstlisting}[breaklines=true]
	> vertebralData <- read.table("VertebralData.2C.txt",header=T,sep=",")
	> usefulData = vertebralData[, c("Incidence", "Tilt", "Angle", "Slope", "Radius", "Degree")]
	> predictors = names(usefulData)
	> means = lapply(usefulData, mean)
	> sd = lapply(usefulData, sd)
	> usefulData = (usefulData - means) / sd
	> usefulData[, "Status"] = vertebralData[, "Status"]
	> summary(usefulData)
	Incidence            Tilt             Angle        
	Min.   :-1.9928   Min.   :-2.4078   Min.   :-2.0443
	1st Qu.:-0.8161   1st Qu.:-0.6870   1st Qu.:-0.8047
	Median :-0.1048   Median :-0.1184   Median :-0.1277
	Mean   : 0.0000   Mean   : 0.0000   Mean   : 0.0000
	3rd Qu.: 0.7183   3rd Qu.: 0.4574   3rd Qu.: 0.5966
	Max.   : 4.0227   Max.   : 3.1862   Max.   : 3.9782
	Slope                Radius              Degree             
	Min.   :-2.20418  Min.   :-3.5922   Min.   :-0.9946
	1st Qu.:-0.71569  1st Qu.:-0.5415   1st Qu.:-0.6574
	Median :-0.04089  Median : 0.0261   Median :-0.3868
	Mean   : 0.00000  Mean   : 0.0000   Mean   : 0.0000
	3rd Qu.: 0.72577  3rd Qu.: 0.5667   3rd Qu.: 0.3991
	Max.   : 5.84632  Max.   : 3.3903   Max.   :10.4435
	Status           
	Abnormal:210     
	Normal  :100  
	> attach(usefulData)
	> glm.mod <- glm(Status~Incidence+Tilt+Angle+Slope+Radius+Degree, data=usefulData, family=binomial)
	> summary(glm.mod)
	
	Call:
	glm(formula = Status ~ Incidence + Tilt + Angle + Slope + Radius + 
	Degree, family = binomial, data = usefulData)
	
	Deviance Residuals: 
	Min       1Q   Median       3Q      Max  
	-2.2678  -0.3639  -0.0289   0.4081   2.7317  
	
	Coefficients:
	Estimate Std. Error z value Pr(>|z|)    
	(Intercept) -3.216e+00  4.803e-01  -6.696 2.14e-11 ***
	Incidence    4.339e+08  6.923e+08   0.627    0.531    
	Tilt        -2.520e+08  4.020e+08  -0.627    0.531    
	Angle        3.329e-01  4.248e-01   0.784    0.433    
	Slope       -3.379e+08  5.392e+08  -0.627    0.531    
	Radius       1.434e+00  3.087e-01   4.645 3.39e-06 ***
	Degree      -6.357e+00  8.770e-01  -7.248 4.23e-13 ***
	---
	Signif. codes:  0 ‘***’ 0.001 ‘**’ 0.01 ‘*’ 0.05 ‘.’ 0.1 ‘ ’ 1
	
	(Dispersion parameter for binomial family taken to be 1)
	
	Null deviance: 389.86  on 309  degrees of freedom
	Residual deviance: 177.87  on 303  degrees of freedom
	AIC: 191.87
	
	Number of Fisher Scoring iterations: 8
	
	> glm.probs <- predict(glm.mod, type="response")
	> contrasts(Status)
	Normal
	Abnormal      0
	Normal        1
	> glm.pred = rep("Abnormal",dim(usefulData)[1])
	> glm.pred[glm.probs >.5] = "Normal"
	> table(glm.pred,Status)
	Status
	glm.pred   Abnormal Normal
	Abnormal      186     23
	Normal         24     77
	> mean(glm.pred==Status)
	[1] 0.8483871
	
	> library(MASS)
	> lda.mod=lda(Status~Incidence+Tilt+Angle+Slope+Radius+Degree, data=usefulData)
	Warning message:
	In lda.default(x, grouping, ...) : variables are collinear
	> summary(lda.mod)
	Length Class  Mode     
	prior    2     -none- numeric  
	counts   2     -none- numeric  
	means   12     -none- numeric  
	scaling  6     -none- numeric  
	lev      2     -none- character
	svd      1     -none- numeric  
	N        1     -none- numeric  
	call     3     -none- call     
	terms    3     terms  call     
	xlevels  0     -none- list     
	> lda.probs <- predict(lda.mod, type="response")
	> lda.pred = unlist(lda.probs[1])
	> table(lda.pred,Status)
	Status
	lda.pred   Abnormal Normal
	Abnormal      193     27
	Normal         17     73
	> mean(lda.pred==Status)
	[1] 0.8580645
	\end{lstlisting}
	Predict the variable Status suing logistic regression and LDA are shown in the code above. The correct rate is used to evaluate these two models. The one using logistic regression has the correct rate 0.8483871. While LDA shows a higher correct rate 0.8580645.\\
	% --------------------------------------------------------------
	%     You don't have to mess with anything below this line.
	% --------------------------------------------------------------
	
\end{document}