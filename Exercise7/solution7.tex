% --------------------------------------------------------------
% This is all preamble stuff that you don't have to worry about.
% Head down to where it says "Start here"
% --------------------------------------------------------------

\documentclass[12pt]{article}

\usepackage[margin=1in]{geometry} 
\usepackage{amsmath,amsthm,amssymb}
\usepackage{color}
%\usepackage{tikz, pgfplots}
\usepackage{graphicx}
\usepackage{epstopdf} %converting to PDF
\usepackage{subcaption}
\usepackage{listings}

\makeatletter

\renewcommand\section{\@startsection {section}{1}{\z@}%
	{-3.5ex \@plus -1ex \@minus -.2ex}%
	{2.3ex \@plus.2ex}%
	{\normalfont\large\bfseries}}% from \Large
\renewcommand\subsection{\@startsection{subsection}{2}{\z@}%
	{-3.25ex\@plus -1ex \@minus -.2ex}%
	{1.5ex \@plus .2ex}%
	{\normalfont\large\bfseries}}% from \large
\makeatother

\begin{document}
	
	% --------------------------------------------------------------
	%                         Start here
	% --------------------------------------------------------------
	
	%\renewcommand{\qedsymbol}{\filledbox}
	
	\title{\textbf{Statistical Learning Methods Exercise \#{7}}\\
	Universit{\'e} de Neuch\^{a}tel}%replace X with the appropriate number
	\author{{Lin Bai, 09935404}} %replace with your name
	
	\maketitle

	%%%%%%%%%%%%%%%%%%%%%%%%%%%%%%%%%%%%%%%%%%%%%%%%%%%%%%%%%%%%%%%%%%%%%%%%%%%%%%%%%%%
	%%%%%%   question 1
	%%%%%%%%%%%%%%%%%%%%%%%%%%%%%%%%%%%%%%%%%%%%%%%%%%%%%%%%%%%%%%%%%%%%%%%%%%%%%%%%%%%
	\section{Solution to question 1}
	Data set preprocessing :\\
	\lstset{language=R}
	\lstset{frame=lines}
	\lstset{label={lst:code_direct}}
	\lstset{basicstyle=\footnotesize\ttfamily}
	\begin{lstlisting}[breaklines=true]
	library(FNN)
	
	# Computer data : need to predict PRP.
	computerData = read.table("ComputerData.txt", header=T)
	
	# Remove the variable model (name), vendor (name) and ERP
	usefulData = computerData
	[, c("MYCT", "MMIN", "MMAX", "CACH", "CGMIN", "CHMAX", "PRP")]
	predictors = names(usefulData)
	summary(usefulData)
	
	# Standardized the values
	means = lapply(usefulData, mean)
	sd = lapply(usefulData, sd)
	usefulData = (usefulData - means) / sd
	summary(usefulData)
	attach(usefulData)
	\end{lstlisting}
	The single regression and multiple regression is shown below.\\
	\lstset{language=R}
	\lstset{frame=lines}
	\lstset{label={lst:code_direct}}
	\lstset{basicstyle=\footnotesize\ttfamily}
	\begin{lstlisting}[breaklines=true]
	library(boot)
	set.seed(123)
	cv.errorSigReg <- rep(0,10)
	cv.SigMse <- rep(0,10)
	for (i in 1:10) {
	  glm.mod <- glm(PRP~poly(MMAX,i), data=usefulData)
	  cv.SigMse[i] <- sum((glm.mod$residuals)^2)
	  cv.errorSigReg[i] <- cv.glm(usefulData,glm.mod,K=10)$delta[2]
	}
	cv.SigMse
	\end{lstlisting}
	The MSE of single regression and multiple regression are:\\	
	\lstset{language=R}
	\lstset{frame=lines}
	\lstset{label={lst:code_direct}}
	\lstset{basicstyle=\footnotesize\ttfamily}
	\begin{lstlisting}[breaklines=true]
	> cv.SigMse
	[1] 53.08657 34.97312 34.72626 34.66615 34.61224 
        34.55119 34.54937 34.52780 34.49340 34.47866
	> cv.MulMse
	[1] 47.12190 38.89017 38.87990 38.05286 37.56264 
	    33.36209 30.32229 26.61521 25.96888 25.94060
	> mean(cv.SigMse)
	[1] 36.46648
	> mean(cv.MulMse)
	[1] 34.27165
	\end{lstlisting}
	The MSE of multiple regression is a little bit better than that of single regression.
	%\begin{figure}[htbp]
	%	\centering
	%	\includegraphics[width=0.75\textwidth]{q1.eps}
	%	\caption{education wage plot}
	%\end{figure}
	%\newpage
	%%%%%%%%%%%%%%%%%%%%%%%%%%%%%%%%%%%%%%%%%%%%%%%%%%%%%%%%%%%%%%%%%%%%%%%%%%%%%%%%%%%
	%%%%%%   question 2
	%%%%%%%%%%%%%%%%%%%%%%%%%%%%%%%%%%%%%%%%%%%%%%%%%%%%%%%%%%%%%%%%%%%%%%%%%%%%%%%%%%%
	\section{Solution to question 2}
	The KNN regression with LOO. If "test" in knn.reg function is set to "NULL", it will
	run with Leaving-One-Out.\\
	\lstset{language=R}
	\lstset{frame=lines}
	\lstset{label={lst:code_direct}}
	\lstset{basicstyle=\footnotesize\ttfamily}
	\begin{lstlisting}[breaklines=true]
	computer.knn = knn.reg(usefulData, test = NULL, y = PRP, k = 2)
	sum(computer.knn$residuals^2)
	\end{lstlisting}
	%%%%%%%%%%%%%%%%%%%%%%%%%%%%%%%%%%%%%%%%%%%%%%%%%%%%%%%%%%%%%%%%%%%%%%%%%%%%%%%%%%%
	%%%%%%   question 3
	%%%%%%%%%%%%%%%%%%%%%%%%%%%%%%%%%%%%%%%%%%%%%%%%%%%%%%%%%%%%%%%%%%%%%%%%%%%%%%%%%%%
	\section{Solution to question 3}
	To get the MSE of knn regression with LOO.\\
	\lstset{language=R}
	\lstset{frame=lines}
	\lstset{label={lst:code_direct}}
	\lstset{basicstyle=\footnotesize\ttfamily}
	\begin{lstlisting}[breaklines=true]
	> sum(computer.knn$residuals^2)
	[1] 26.79073
	\end{lstlisting}
	Obviously, knn has much less MSE comparing to regression.
	%%%%%%%%%%%%%%%%%%%%%%%%%%%%%%%%%%%%%%%%%%%%%%%%%%%%%%%%%%%%%%%%%%%%%%%%%%%%%%%%%%%
	%%%%%%   question 4
	%%%%%%%%%%%%%%%%%%%%%%%%%%%%%%%%%%%%%%%%%%%%%%%%%%%%%%%%%%%%%%%%%%%%%%%%%%%%%%%%%%%
	\section{Solution to question 4}
	From the test result, knn is better comparing to single regression and multiple regression.\\
	\textbf{Advantage:}\\
	knn: no training needed.non-linear model, more precise.\\
	regression: more robust.\\
	\textbf{Disadvantage:}\\
	knn: $K$ value needs tunning.\\
	regression: need training, less precise.\\
	It's hard to say in any case which one is better. However, in this computer data case, 
	\textbf{knn} is abetter choice.\\
	% --------------------------------------------------------------
	%     You don't have to mess with anything below this line.
	% --------------------------------------------------------------
	
\end{document}