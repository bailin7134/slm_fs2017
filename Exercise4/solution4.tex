% --------------------------------------------------------------
% This is all preamble stuff that you don't have to worry about.
% Head down to where it says "Start here"
% --------------------------------------------------------------

\documentclass[12pt]{article}

\usepackage[margin=1in]{geometry} 
\usepackage{amsmath,amsthm,amssymb}
\usepackage{color}
%\usepackage{tikz, pgfplots}
\usepackage{graphicx}
\usepackage{epstopdf} %converting to PDF
\usepackage{subcaption}
\usepackage{listings}

\makeatletter

\renewcommand\section{\@startsection {section}{1}{\z@}%
	{-3.5ex \@plus -1ex \@minus -.2ex}%
	{2.3ex \@plus.2ex}%
	{\normalfont\large\bfseries}}% from \Large
\renewcommand\subsection{\@startsection{subsection}{2}{\z@}%
	{-3.25ex\@plus -1ex \@minus -.2ex}%
	{1.5ex \@plus .2ex}%
	{\normalfont\large\bfseries}}% from \large
\makeatother

\begin{document}
	
	% --------------------------------------------------------------
	%                         Start here
	% --------------------------------------------------------------
	
	%\renewcommand{\qedsymbol}{\filledbox}
	
	\title{\textbf{Statistical Learning Methods Exercise \#{4}}\\
	Universit{\'e} de Neuch\^{a}tel}%replace X with the appropriate number
	\author{{Lin Bai, 09935404}} %replace with your name
	
	\maketitle

	%%%%%%%%%%%%%%%%%%%%%%%%%%%%%%%%%%%%%%%%%%%%%%%%%%%%%%%%%%%%%%%%%%%%%%%%%%%%%%%%%%%
	%%%%%%   question 1
	%%%%%%%%%%%%%%%%%%%%%%%%%%%%%%%%%%%%%%%%%%%%%%%%%%%%%%%%%%%%%%%%%%%%%%%%%%%%%%%%%%%
	\section{Solution to question 1}
	\begin{figure}[htbp]
		\centering
		\includegraphics[width=0.75\textwidth]{q1.eps}
		\caption{education wage plot}
	\end{figure}
	\noindent
	conclusion:\\
	the wage is a linear function of eduction for both male and female.\\
	the linear model for male is: wage = education * 398.5 + 24\\
	the linear model for female is:	wage = eduction * 397.5 - 563.5\\
	%%%%%%%%%%%%%%%%%%%%%%%%%%%%%%%%%%%%%%%%%%%%%%%%%%%%%%%%%%%%%%%%%%%%%%%%%%%%%%%%%%%
	%%%%%%   question 2
	%%%%%%%%%%%%%%%%%%%%%%%%%%%%%%%%%%%%%%%%%%%%%%%%%%%%%%%%%%%%%%%%%%%%%%%%%%%%%%%%%%%
	\section{Solution to question 2}
	Vendor name and model names have no relation to the system performance, therefore both of them are not able to be the predictive values.\\
	%%%%%%%%%%%%%%%%%%%%%%%%%%%%%%%%%%%%%%%%%%%%%%%%%%%%%%%%%%%%%%%%%%%%%%%%%%%%%%%%%%%
	%%%%%%   question 3
	%%%%%%%%%%%%%%%%%%%%%%%%%%%%%%%%%%%%%%%%%%%%%%%%%%%%%%%%%%%%%%%%%%%%%%%%%%%%%%%%%%%
	\section{Solution to question 3}
	From the Summary Statistics, MMAX has the maximum PRP Correlation value. Therefore, MMAX is the variable which is chosen.\\
	\begin{table}[htbp]
		\centering
		\caption{table for question 3}
		\label{q3_table}
		\begin{tabular}{|l|l|l|}
			\hline
			      & Multiple R-squared & F-statistic\\
			\hline
			MYTC  & 0.09431            & 21.56 \\
			\hline
			MMIN  & 0.6319             & 355.4 \\
			\hline
			MMAX  & 0.7448             & 604.1 \\
			\hline
			CACH  & 0.4391             & 162 \\
			\hline
			CGMIN & 0.3708             & 122 \\
			\hline
			CHMAX & 0.3663             & 119.6\\
			\hline
		\end{tabular}
	\end{table}
	\noindent
	\\
	From the multiple R-squared and F-statistic result, MMAX also has the biggest value.\\
	The confident interval is (-49.773, -18.226)
	\lstset{language=R}
	\lstset{frame=lines}
	\lstset{label={lst:code_direct}}
	\lstset{basicstyle=\footnotesize\ttfamily}
	\begin{lstlisting}[breaklines=true]
	> confint(mld)
	              2.5 %    97.5 %
	(Intercept) -49.77265640 -18.22582429
	MMAX          0.01088673   0.01278561
	\end{lstlisting}
	\newpage
	%%%%%%%%%%%%%%%%%%%%%%%%%%%%%%%%%%%%%%%%%%%%%%%%%%%%%%%%%%%%%%%%%%%%%%%%%%%%%%%%%%%
	%%%%%%   question 4
	%%%%%%%%%%%%%%%%%%%%%%%%%%%%%%%%%%%%%%%%%%%%%%%%%%%%%%%%%%%%%%%%%%%%%%%%%%%%%%%%%%%
	\section{Solution to question 4}
	\begin{figure}[htbp]
		\centering
		\includegraphics[width=0.75\textwidth]{q4.eps}
		\caption{plot for question 4}
	\end{figure}
	%\noindent
	%%%%%%%%%%%%%%%%%%%%%%%%%%%%%%%%%%%%%%%%%%%%%%%%%%%%%%%%%%%%%%%%%%%%%%%%%%%%%%%%%%%
	%%%%%%   question 5
	%%%%%%%%%%%%%%%%%%%%%%%%%%%%%%%%%%%%%%%%%%%%%%%%%%%%%%%%%%%%%%%%%%%%%%%%%%%%%%%%%%%
	\section{Solution to question 5}
	Among all the variables, $name$ and $origin$ are not able to used to explain $mpg$, since they contribute nothing to the $mpg$.\\
	The rest ones, like $cylinders$, $displacement$, $horsepower$, $weight$, $acceleration$ and $year$ have relation to the $mpg$.\\
	%%%%%%%%%%%%%%%%%%%%%%%%%%%%%%%%%%%%%%%%%%%%%%%%%%%%%%%%%%%%%%%%%%%%%%%%%%%%%%%%%%%
	%%%%%%   question 6
	%%%%%%%%%%%%%%%%%%%%%%%%%%%%%%%%%%%%%%%%%%%%%%%%%%%%%%%%%%%%%%%%%%%%%%%%%%%%%%%%%%%
	\section{Solution to question 6}
	To choose the variable, linear model between $mpg$ and all other variables are built.
	\begin{table}[htbp]
		\centering
		\caption{table for question 6}
		\label{q3_table}
		\begin{tabular}{|l|l|l|}
			\hline
			             & Multiple R-squared & F-statistic\\
			\hline
			cylinders    & 0.6012             & 597.1 \\
			\hline
			displacement & 0.6467             & 725 \\
			\hline
			horsepower   & 0.6059             & 599.7 \\
			\hline
			weight       & 0.6918             & 888.9 \\
			\hline
			acceleration & 0.1766             & 84.96 \\
			\hline
			year         & 0.3356             & 200 \\
			\hline
		\end{tabular}
	\end{table}
	\noindent
	\\
	$weight$ has the largest value in both multiply R-squared and F-statistic. So $weight$ is the one can be chosen to explain $mpg$.\\
	\lstset{language=R}
	\lstset{frame=lines}
	\lstset{label={lst:code_direct}}
	\lstset{basicstyle=\footnotesize\ttfamily}
	\begin{lstlisting}[breaklines=true]
	> confint(summaryMld)
	                2.5 %       97.5 %
	(Intercept) 44.753934082 47.880794759
	weight      -0.008182822 -0.007170398
	\end{lstlisting}
	%%%%%%%%%%%%%%%%%%%%%%%%%%%%%%%%%%%%%%%%%%%%%%%%%%%%%%%%%%%%%%%%%%%%%%%%%%%%%%%%%%%
	%%%%%%   question 7
	%%%%%%%%%%%%%%%%%%%%%%%%%%%%%%%%%%%%%%%%%%%%%%%%%%%%%%%%%%%%%%%%%%%%%%%%%%%%%%%%%%%
	\section{Solution to question 7}
	\begin{figure}[htbp]
		\centering
		\includegraphics[width=0.74\textwidth]{q7.eps}
		\caption{plot for question 7}
	\end{figure}
	% --------------------------------------------------------------
	%     You don't have to mess with anything below this line.
	% --------------------------------------------------------------
	
\end{document}